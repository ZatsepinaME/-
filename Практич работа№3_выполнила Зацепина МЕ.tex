\documentclass[a4paper,11pt]{article}
\usepackage[T2A]{fontenc}
\usepackage[utf8]{inputenc}
\usepackage{color}
%\usepackage{verbatim}
\usepackage{amsmath}
\usepackage[european,cuteinductors,smartlabels]{circuitikz}

\usepackage[english, russian]{babel}

\DeclareMathOperator{\Coker}{Coker}
\DeclareMathOperator{\Ker}{Ker}
%\usepackage{pgfplots}
\usetikzlibrary{positioning}
\usetikzlibrary{quotes,angles}
\usetikzlibrary{arrows,shapes}
\usepackage{xifthen}

%\usepackage[active,tightpage]{preview}
%\setlength\PreviewBorder{0pt}
\usepackage{pgf,tikz}
\usepackage{pgfplots}

\usetikzlibrary{backgrounds}
%\usetikzlibrary{decorations}

\title{Практическая работа №3}
\author{Зацепина М.Е.}
%конец преамбулы

\begin{document}
\maketitle 
\textbf {Задание:\,\,\,Построение графика функции}\\[11pt]
$x^3 - 4x^2 -3x - 6 = 0$
\definecolor{darkgray}{rgb}{0.25,0.25,0.25}
%\definecolor{lightgray}{rgb}{0.75,0.75,0.75}


  \maketitle 
Решение показано на листе 2:
		
%\pgfplotsset{compat=1.9};
\begin{tikzpicture} [descr/.style={fill=white},text height=3ex, text depth=1ex]
\begin{scope}[scale=0.5]

%%Axis
%%\draw [<->] (0,15) node (yaxis) [left] {$Y$} -- (0,-15) -- (-15,0) -- (15,0) node (xaxis) [below] {$X$};
%%                                                                                    --(0,0) node[below left] {(0,0)}
    \draw[loosely dotted] (-9,-25) grid (9,13);
   


 
% Macros for cst. They have to be redefined each time. See inside document
%\newcommand{\cA}{0.2}%	Cste . fct


%LaTeX Macro for drawing fct with pgf/tikz. Define once, use many!
%\newcommand{\FctAss}

 % \pgfmathparse{0.1+\cA*1.1} \pgfmathresult \let\maxY\pgfmathresult% evaluate maxY 
%  \pgfmathparse{-0.1-\cA*1.1} \pgfmathresult \let\minY\pgfmathresult% evaluate minY
  %   \pgfmathparse{\maxY < 1} \pgfmathresult \let\BmaxY\pgfmathresult% What if maxY < 1? Then set Boolean to 1
    %   \ifthenelse{\equal{\BmaxY}{3.0}}{%
     %  \pgfmathparse{1.2} \pgfmathresult \let\maxY\pgfmathresult% Correct maxY to have correct graph
  %     }{}
   %  \pgfmathparse{\minY > 0} \pgfmathresult \let\BminY\pgfmathresult% What if minY > 0? Then set Boolean to 1
  %     \ifthenelse{\equal{\BminY}{3.0}}{%
    %   \pgfmathparse{0} \pgfmathresult \let\minY\pgfmathresult% Correct minY to have correct graph
       
%        DRAW the graph of the function from here on
%   \draw[very thin,color=gray] (-0.5,\minY) grid (3.9,\maxY);% GRID use minY & maxY
%    \draw[->] (-0.2,0) -- (8.2,0) node[right] {$x$};
%    \draw[->] (0,\minY) -- (0,\maxY) node[above] {$f(x)$};% y axis use minY & maxY too
%    \draw[smooth,samples=200,color=blue] plot function{\cA^3-4*\cA^2-3*\cA-6} 
  %      node[right] {$f(x) =\cA^3-4*(\cA^2)-3*\cA-6$};
% units for cartesian reference frame
   


   %%   \foreach \x in {0,1}
   %%   \draw (\x cm,3pt) -- (\x cm,-3pt)
   %%         node[anchor=north,xshift=-0.15cm] {$\x$};
    %%  \foreach \y/\ytext in {0,4}
    %%  \draw (2pt,\y cm) -- (-2pt,\y cm) node[anchor=east] {$\ytext$};


   % [line cap=round,line join=round,x=2cm,y=2cm,
     
     %using the decoration 'brace' (=a curly brace as path replacement)
    % decoration={brace,amplitude=2pt}]
%main layer
%creating the grid
 % \draw [color=lightgray,dash pattern=on 1pt off 1pt, xstep=2cm,ystep=2cm]
                                                % (-0.1,-0.1) grid (2.3,2.3);
%creating the ticks and xy-axis nodes
  \draw[-latex,color=darkgray,thin] (-10,0) -- (10,0)  node (xaxis) [below] {$X$};
   \foreach \x in {-1,1}
   \draw[shift={(\x,0)},color=darkgray,thin] (0pt,3pt) -- (0pt,-3pt)
                                   node[below left] {\footnotesize $\x$};
  \draw[-latex,color=darkgray] (0,-25) -- (0,14)  node (yaxis) [left] {$Y$};
      \foreach \y in {-2,2}
      \draw[shift={(0,\y)},color=darkgray,thin] (3pt,0pt) -- (-3pt,0pt)
                                    node[left] {\footnotesize $\y$};
  \draw[color=black] (-1pt,-12pt) node[left] {\footnotesize $0$};
%some function
  \draw[smooth,samples=10,domain=-5:5.0] 

   plot(\x,{\x^3 - 4*\x^2 -3*\x-6});
%\draw[thick]     node[above right] {$y=x^3 - 4x^2 -3x-6$};
\draw (4.5cm,160pt) node[above]
        {$y=x^3 - 4x^2 -3x-6$};

% END of macro
%\begin{preview}
% And now use it!
%    \FctAss{}
    
% Change the parameters
 %   \renewcommand{\cA}{2}
   
% WITHOUT rewriting the code for the picture    
%    \FctAss{}
% 
% And do it again    
  %  \renewcommand{\cA}{1.2}
    
    
    
%\end{preview}







%    \foreach \x/\xtext in {1/1, 2/2, 3/3}
   
% \draw[shift={(\x,0)}] (0pt,2pt) -- (0pt,-2pt) node[below] {$\xtext$};
%    \foreach \y/\ytext in {1/1, 2/2, 3/3, 4/4}
%   \draw[shift={(0,\y)}] (2pt,0pt) -- (-2pt,0pt) node[left] {$\ytext$};


%\newcommand{\xb}{-1}
%\newcommand{\xa}{1}
%\begin{scope}[scale=0.8]
%\draw[thin, ->] (-6,0) -- (6.5,0) node[right] {$X$};
%\draw[thin, ->] (0,-2.5) -- (0,4) node[left] {$Y$};
%\foreach \x\xtext in  {-5/-5,5/5,{\xb}/\xb,{\xa}\xa}}  
 % \draw (\x,0.1) -- (\x,-0.1) node[below] {$\xtext$};

 %\draw[domain=-5:5, help lines, smooth, red]
   %plot ({\x},{0.2*(\x-\xa)*(\x-\xb)});


%\newcommand{\xb}{-2}
%\newcommand{\xa}{1}
%\draw[thin, ->] (-1,0) -- (2.5,0) node[right] {$X$};
%\draw[thin, ->] (0,-6) -- (0,1.5) node[left] {$Y$};
%\foreach \x\xtext in
%{-5/-5,5/5,{\xb}/\xb,{\xa}/\xb} %

%\draw (\x,0.1) -- (\x,-0.1) node[below] {$\xtext$};
%\draw[domain=-5:5, help lines, smooth, red]
%plot ({\x},{0.2*(\x-\xa)*(\x-\xb)});

%\begin{axis}[	title = График фнукции,
	%xlabel = {$x$},
	%ylabel = {$y$},
	%minor tick num = 1 ]
%\addplot[blue] {x^3 - 0.5*x^2 -3.5*x-3};
%\end{axis}

\end{scope}
\end{tikzpicture}




\end{document}
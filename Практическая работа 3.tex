\documentclass{article}
\usepackage[T2A]{fontenc}
\usepackage[utf8]{inputenc}
\usepackage[english,russian]{babel}
\usepackage[european,cuteinductors,smartlabels]{circuitikz}
\usepackage{amssymb,amsmath,amsfonts,latexsym,mathtext}
\usepackage{tikz}
\usepackage{pgffor}
%\usepackage{icomma}
%построение графика
\usepackage{pgfplots}
\pgfplotsset{compat=1.9}%используемую версию ( указана версия 1.9)

\title {Практическая работа №3}
\author{Зацепина МЕ}
%определение стиля
\pgfplotsset{model/.style = {blue, samples = 50}, title = График нелинейной функции}% точек разбиения =50
\begin{document}
	\maketitle
	\textbf{Задание: Построение графика функции}\\ [11pt]
	$x^3+0.5*x^2-3.5*x-3=0$\\\\
	\maketitle
	Решение:\\
	\\
	\maketitle\\
	\begin{tikzpicture}
	
		%постр графика
		\begin{axis}[
		height = 0.7\paperheight, 
		width = 0.7\paperwidth,
		xmin = -15, ymax = 120,
		legend pos = north west,
		xlabel = {$x$},
		ylabel = {$y$},
		grid=major]
		\addplot[thick,model,mark=*,mark options={scale= 0.5, fill = black, draw = green}] {x^3+x^2*0.5-x*3.5-3};
	%\addplot[thick,model,mark=*,mark options={fill = black, draw = green}] {(x^3)+((x^2)*0.5)-x*3.5-3};
		\addplot[thick,mark options={scale= 0.5}] {10.5*(\x)-21}; % y (х=2)
	%	\addplot[thick,mark options={scale= 0.5}] {13*(\x)-15.5}; %y' (x=2)
	%	\addplot[smooth, red,mark options={scale= 0.5}] {0.5*(\x)-5.5}; % y (х=1)
	%	\addplot[smooth, red,mark options={scale= 0.5}] {7*(\x)-6.5};   %y' (x=1)
		
	
	
			\foreach \x in {0,2,...,5}
	{
		%\draw[domain=1:30, help lines, scale= 0.5, smooth, red]	plot ({\x},{7*(\x)-6.5});% построенная линия -касательная не касается графика, а проходит на расстоянии 
		\draw[domain=-1:20, help lines, scale= 0.5, smooth, red]	plot ({\x},{13*(\x)-15.5}); %% построенная линия -касательная не касается графика, а проходит на расстоянии
	}
	\legend{ $y=x^3+x^2*0.5-x*3.5-3$};
	\end{axis}
	\end{tikzpicture}
	
	
\end{document}
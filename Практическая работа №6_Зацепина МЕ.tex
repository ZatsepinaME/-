\documentclass{article}
\usepackage[T2A]{fontenc}
\usepackage[utf8]{inputenc}
\usepackage[english,russian]{babel}
\usepackage{tikz}
\usepackage[european,cuteinductors,smartlabels]{circuitikz}
\usepackage{tikz}
\usepackage[european,cuteinductors,smartlabels]{circuitikz}
\usepackage{booktabs}
\usepackage{multicol}
\usepackage{multirow}
\usepackage{amsmath}

\title{Практическая работа №5}
\author{студент: Зацепина М.Е.    группы № 8871   }
% Конец преамбулы
\begin{document}
\maketitle

Задание: Нарисовать электрическую схему.

Решение: \\[20pt]

	
	\begin{circuitikz}
		\tiny
		\draw (1.5,0) -- (1.5,-1);
		\draw (0,-3) to [C, l_={$C$},*-*] (3,-3);
		\draw (0,-3) to [R, l={$R$},*-*] (1.5,-1);
		\draw (1.5,-1) to [L, l={$L_1$},*-*] (3,-3);
		
		\draw (1.5,0) -- (8,0);
		\draw (3,-3) -- (3,-4);
		\draw (3,-4) -- (7,-4);
		\draw (0,-3) -- (0,-5);
		\draw (0,-5) -- (9,-5);
		\draw (8,0) to [L, l={$L_2$},*-*] (8,-2);
		\draw (8,-2)  to [L, l,*-*] (7,-4);
		\draw (8,-2) to [L, l={$L_4$},*-*] (9,-4);
		\draw (9,-4) -- (9,-5);
	%	\rotatebox[$L_3$]{60}{$L_3$} node [left] at (7.2,-2.5);
		\draw[thin,dotted] (7,-4) -- (8,-2) node[pos=0.68,sloped, above]  {$L_3$};
	\end{circuitikz}



\end{document}
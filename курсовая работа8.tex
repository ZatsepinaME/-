

%\documentclass[a4paper,russian]{article}
%\usepackage[T2A]{fontenc}
%\usepackage[T2A]{fontenc}
%\usepackage[utf8x]{inputenc}
%\usepackage{pscyr}
%\usepackage[english, russian]{babel}




\documentclass[russian,utf8,nocolumnxxxi,nocolumnxxxii]{eskdtext}
	
\usepackage[T2A]{fontenc}
\usepackage[utf8]{inputenc}
\usepackage[english,russian]{babel}
\usepackage[european,cuteinductors,smartlabels]{circuitikz}
\usepackage{amssymb,amsmath,amsfonts,latexsym,mathtext}
\usepackage[utf8]{inputenc}
%\usepackage[utf8x]{inputenc}
%\usepackage[english,ukrainian,russian]{babel}
\usepackage{amssymb,amsmath}
\usepackage{amsmath,amsfonts,amssymb,amsthm,mathtools} % AMS
\usepackage{tikz}
\usetikzlibrary{arrows,automata,shapes,calc}
%usepackage[european,cuteinductors,smartlabels]{circuitikz}
\usepackage{pgffor}
%\usetikzlibrary{datavisualization}
%\usetikzlibrary{datavisualization.formats.functions}
\usepackage{pgfplots}
%\pgfplotsset{compat=newest}
\pgfplotsset{compat=1.9}%используемую версию ( указана версия 1.9)
%usepackage{circuitikz}
\usepackage{siunitx}
%usepackage[american,cuteinductors,smartlabels]{circuitikz}
\usepackage{indentfirst} % Красная строка
\usepackage{graphicx}
\graphicspath{{pictures/}}
\DeclareGraphicsExtensions{.pdf,.png,.jpg}
%\usepackage[backend=biber]{biblatex}
%\addbibresource{error_estimation_otchet.bib}
\usepackage{mathtools}
%\mathtoolsset{showonlyrefs}
%\mathtoolsset{showonlyrefs=true} % Показывать номера только у тех формул, на которые есть \eqref{} в тексте.
\usepackage[]{hyperref}
\hypersetup{
	colorlinks=true,
}
\usepackage{multirow}
\usepackage{textcomp}
\newcommand{\No}{\textnumero}

\ESKDdepartment{Федеральное агентство по образованию}
\ESKDcompany{Санкт-Петербургский государственный электротехнический университет "ЛЭТИ"}
\ESKDtitle{Пояснительная записка по дисциплине "Информатика"}
\ESKDdocName{Вариант №7}
	\ESKDsignature{Курсовая работа}
	\ESKDauthor{Зацепина~МЕ}
	\ESKDchecker{Прокшин~АН}
	
   \begin{document}
   %	\maketitle
 	
    \maketitle
    \tableofcontents
    \newpage
     \section{Вступление}
    \itshape{\Large Цель курсовой работы:}\\[4pt]
    \rmfamily{уметь применять персональный компьютер и математические пакеты прикладных программ в инженерной деятельности}\\[18pt]
    ${\,\,\,\,\,\,\,\,\,\,\,\,\,\,\,\,\,}$\itshape{\Large Тема курсовой работы:}\\[4pt]
    \rmfamily{решение математических задач с использованием математического пакета "Scilab" и "Smath".}\\
    
    \newpage
    \section{ Основная часть}
     \subsection{Задание на курсовую работу}
     \upshape{\normalsize 1. Даны функции:}\\[4pt]
      {$f{(x)}$ = $\sqrt{3}sin(x)$ + cos(x), {\,}$и$ \\
     {g{(x)} = cos(2{x} + ${\pi}/3$)-1}\\[7pt]
    a)	Решить уравнение ${f{(x)}$ = ${g(x).}$ \\
    b)	Исследовать функцию $h{(x)} = f{(x)} – g{(x)}$ на промежутке [0 ; (5$\pi)$/6].\\ [10pt]
     %\upshape
     2. Найти коэффициенты кубического сплайна, интерполирующего данные, представленные в векторах $\vec{V}$_x и $\vec{V}$_y.}\,
     $Построить на одном графике функцию $f(x)$ и функцию $f_1(x)$, полученную после нахождения коэффициентов кубического сплайна.$\\
     $Представить графическое изображение результатов интерполяции исходных данных различными методами с использованием встроенных функций $cspline(V_x,V_y)$, $pspline(V_x,V_y)$, $lspline(V_x,V_y)$ и $interp(V_k,V_x,V_y,x)$.\\[4pt]
     \upshape{\normalsize 3. Решить задачу оптимального распределения неоднородных ресурсов.}\\
     Исходные данные представлены в таблице 1.
    
     \begin{table}[ht]
     	\caption{ }\label{tab:tab1}
     	\centering%центрируем таблицу	
     	 \begin{tabular}{|1|1|1|1|1|1|}
     		%\toprule
     		\hline
     		%	\multirow{2}{*}{Используемые ресурсы, $a_i$}&
     	%	\multicolumn{4}{|c|}{Изготавливаемые изделия}& 
     	%	\multirow{1}{c|}{Наличие ресурсов, $a_i$}\\
     		 		%\cmidrule{1-5}&
     	%	\multicolumn{1}{|c|}{И_1}&
     	%	\multicolumn{1}{|c|}{И_2}&
     	%	\multicolumn{1}{|c|}{И_3}&
     	%	\multicolumn{1}{|c|}{И_4}&\\
     		% \hline
     	%	\midrule 
     	Используемые ресурсы, {$a_i$}& \multicolumn{4}{|c|}{Изготавливаемые изделия}& Наличие ресурсов,{$a_i$}\\
      & \multicolumn{1}{|c|}{И_1}&
      \multicolumn{1}{|c|}{И_2}&
      \multicolumn{1}{|c|}{И_3}&
      \multicolumn{1}{|c|}{И_4}& \\
     		\hline
     		Трудовые & 2 & 4 &  2 & 9& 20 \\
     		\hline
     		Материальные & 5 & 5 &  5 & 6 & 10\\
     		\hline
     		Финансовые & 5 & 6 &  4 & 8& 30\\
     		\hline
     		Прибыль & 25 & 45 & 60 & 20& \\ 
     		%\bottomrule
     		\hline
     	\end{tabular}
     	
     \end{table} 
     
     \newpage
     \subsection{Решение уравнения и исследование функции}
     	a) {$f{(x)} = g{(x)}$}\\
    {\sqrt{3}}sin(x) + cos(x)= cos(2∙{x} + \frac{\pi}{3})-1\\[7pt]
     	
     	\begin{tabular}{1|1}
     	$\sqrt{3}sin(x) + cos(x)=0$ & $cos(2∙{x} +\frac{\pi}{3})-1=0$\\
     $\sqrt{3}tg(x) + 1=0 & $2∙{x} + \frac{\pi}{3}=arccos{(1)}}$\\
     $tg(x) =-\frac{1}{\sqrt{3}}=-{\frac{\sqrt{3}}{3}}$ &
    $2∙{x} + \frac{\pi}{3}=0$ \\
        & $2∙{x} = -\frac{\pi}{3}$  \\
    $x =-\frac{\pi}{6}+2{\pi}{k},$ & ${x} =-\frac{\pi}{6}+2{\pi}{k},$\\
где\, $ {k}\, {\in}\, {Z}. & где $ {k} \in {Z}.
     	\end{tabular}\\[35pt]
     	
     		$ b)$	Исследовать функцию $h{(x)}$ = $f{(x)}$ – $g{(x)}$ на промежутке [0 ; $\frac{5{\pi}}{6}$].$\\
     
     \textbf{ 1.Область определения функции}\\
      Выражение имеет смысл при любом значении ${x}$ на интервале [0 ; $\frac{5{\pi}}{6}$].\\
      
       \textbf{2.Четность, нечетность функции}\\[10pt]%\label{}% 
       Функция четная, если $y(-x) = y(x)$. \, Функция нечетная, если $y(-x) = -y(x)$. 
       \begin{align*}
       h(x)$=$\sqrt{3}sin(x) + cos(x)$–$(cos(2{x} + {\frac{\pi}{3})-1})
       \end{align*}
       \begin{equation}\label{eq:num1}
       \begin{aligned}
       {h(x)}=\sqrt{3}sin(x)+cos(x)$–$cos(2{x})cos{\frac{\pi}{3}}+sin(2{x})sin{\frac{\pi}{3}}+1})
    \end{aligned}
    \end{equation}\\
    %\begin{equation}
    \begin{aligned}
    	\begin{multiline}
    		{h(-x)}=\sqrt{3}sin(-x)+cos(-x)-cos({-2x})cos{\frac{\pi}{3}}-sin({-2x})sin{\frac{\pi}{3}}+1}
    \end{multiline}
\end{aligned}
%\end{equation}
\begin{equation}\label{eq:num2}
\begin{aligned}
{h(-x)}=-\sqrt{3}sin(x)+cos(x)$–$cos({2x})cos{\frac{\pi}{3}}-sin({2x})sin{\frac{\pi}{3}}+1}
\end{aligned}
\end{equation}
\textbf{Таким образом,} ${h(-x)}\neq {h(x)},$ и ${h(-x)}\neq {-h(x)},$ следовательно функция ${h(x)}$ не обладает свойствами четности и нечетности.

\newpage
      \textbf{3.Нули функции}\\
      \begin{equation}\label{eq:num3}
         \begin{aligned}
         h{(x)} =0
         \end{aligned}
      \end{equation}
      Полученные из уравнения \eqref{eq:num3} значения ${x}$ - это точки пересечения функции {$h(x)$} с осью ОХ.\\
      \begin{equation}\label{eq:num4}
      \begin{aligned}
      $\sqrt{3}$sin(x) + cos(x)$–$(cos(2{x} + {$\frac${$\pi}${3})-1})=0
      \end{aligned}
      \end{equation}
    \\[8pt]
    \left{  
      \begin{equation}\label{eq:num5}
      	      		\begin{aligned}
      	\sqrt{3}sin(x)+cos(x)={2}$\sqrt{3}sin(\frac{x}{2})cos(\frac{x}{2})$+$cos^2(\frac{x}{2})$-$sin^2(\frac{x}{2}) \hspace{153}   
      	\end{aligned}
       	\end{equation}
    }\\[10pt]
    Разделим выражение \eqref{eq:num5} на {cos^2({\frac{x}{2}})}.\\[10pt]
    \left{  
    	\begin{equation}\label{eq:num6}
    	\begin{aligned}
    	\sqrt{3}sin(x)+cos(x)={2}$\sqrt{3}tg(\frac{x}{2})+1-$tg^2(\frac{x}{2}) \hspace{230}   
    	\end{aligned}
    	\end{equation}
    }
       \left{  
      	\begin{equation}\label{eq:num7}
      \begin{multiline}
      \begin{aligned}
      cos(2{x}&+\frac{\pi}{3})-1=
      -(1-cos(2{x}+{\frac{\pi}{3}}))=
      -2sin^2({x}+{\frac{\pi}{6}})=\\=
     	-2&({\frac{\sqrt{3}}{2}}sin({x})+{\frac{1}{2}}cos({x}))^2=\\=
     	-2&({\frac{\sqrt{3}}{2}}2sin({\frac{x}{2}})cos({\frac{x}{2}})+{\frac{1}{2}}cos^2({\frac{x}{2}})-{\frac{1}{2}}sin^2({\frac{x}{2}}))^2=\\=
     	-2&({\sqrt{3}}sin({\frac{x}{2}})cos({\frac{x}{2}})+{\frac{1}{2}}cos^2({\frac{x}{2}})-{\frac{1}{2}}sin^2({\frac{x}{2}}))^2\hspace{185}
      \end{aligned}
      \end{multiline}
      \end{equation}
    }
      \\[10pt]
      Разделим выражение \eqref{eq:num7} на {cos^2({\frac{x}{2}})}:\\[8pt]
      \left{  
      	\begin{equation}\label{eq:num8}
      	\begin{aligned}
      	cos(2{x}+\frac{\pi}{3})-1=-2({\sqrt{3}}tg({\frac{x}{2}})+{\frac{1}{2}}-{\frac{1}{2}}tg^2({\frac{x}{2}}))^2\hspace{190}
      	\end{aligned}
      	\end{equation}
     	\\[8pt]
       Тогда, подставляя выражения \eqref{eq:num7} и \eqref{eq:num8} в \eqref{eq:num4} :
       \begin{align}\label{eq:num9}
       {2}$\sqrt{3}tg(\frac{x}{2})+1-$tg^2(\frac{x}{2})+2({\sqrt{3}}tg({\frac{x}{2}})+{\frac{1}{2}}-{\frac{1}{2}}tg^2({\frac{x}{2}}))^2=0
       \end{align}
       Заменим $tg(\frac{x}{2})=t.$
       \begin{align}\label{eq:num10}
       (2\sqrt{3}t+1-t^2)+2({\sqrt{3}}t+{\frac{1}{2}}-{\frac{1}{2}}t^2)^2=0
       \end{align}
       \begin{align}\label{eq:num11}
       2(\sqrt{3}t+{\frac{1}{2}}-{\frac{1}{2}}t^2)+2({\sqrt{3}}t+{\frac{1}{2}}-{\frac{1}{2}}t^2)^2=0
       \end{align}
       \begin{align}\label{eq:num12}
       (\sqrt{3}t+{\frac{1}{2}}-{\frac{1}{2}}t^2)(2+2({\sqrt{3}}t+{\frac{1}{2}}-{\frac{1}{2}}t^2)=0
       \end{align}
       \begin{align*}
       (\sqrt{3}t+{\frac{1}{2}}-{\frac{1}{2}}t^2)(2+2({\sqrt{3}}t+{\frac{1}{2}}-{\frac{1}{2}}t^2)=0
       \end{align*}
       \begin{align*}
       (\sqrt{3}t+{\frac{1}{2}}-{\frac{1}{2}}t^2)(2+2{\sqrt{3}}t+1-t^2)=0
       \end{align*}
    \begin{align}\label{eq:num13}
    (\sqrt{3}t+{\frac{1}{2}}-{\frac{1}{2}}t^2)(3+2{\sqrt{3}}t-t^2)=0
    \end{align}
    Найдем корни уравнения \eqref{eq:num13}, для этого решим систему:\\
    \begin{equation}\label{eq:num14}
     \begin{align}
    \left\{
    \begin{aligned}
    -2\sqrt{3}t-1+t^2&=0\\
    -3-2{\sqrt{3}}t+t^2&=0
    \end{aligned}\right
    \end{align}
        \end{equation}
     Корни уравнений:\\
     \begin{equation}\label{eq:num15}
     \begin{aligned}
     \left\{
     \begin{aligned}
     {t_1}&={\sqrt{3}+{\sqrt{2}}\\
     {t_2}&={\sqrt{3}}-{\sqrt{2}}\\
     {t_3}&={\sqrt{3}}+{\sqrt{6}}\\
     {t_4}&={\sqrt{3}}-{\sqrt{6}}
     \end{aligned}\right
     \end{aligned}
     \end{equation}    \\  
      \begin{equation}\label{eq:num16}
      \begin{aligned}
      \left\{
      \begin{aligned}
      {tg{\frac{x}{2}&={\sqrt{3}+{}\sqrt{2}}\\
      	{tg{\frac{x}{2}&={\sqrt{3}}-{\sqrt{2}}\\
      {tg{\frac{x}{2}&={\sqrt{3}}+{\sqrt{6}}\\
      {tg{\frac{x}{2}&={\sqrt{3}}-{\sqrt{6}}
    \end{aligned}\right
    \end{aligned}
    \end{equation}\\
    \begin{equation}\label{eq:num17}
    \begin{aligned}
    \left\{
    \begin{aligned}
    {x}&=2arctg({\sqrt{3}+\sqrt{2}}); \,{x}\in[{-\frac{\pi}{2}};0]\\
    	{x}&=2arctg({\sqrt{3}-\sqrt{2}}); \,{x}\in[{-\frac{\pi}{2}};0]\\	{x}&=2arctg({\sqrt{3}+\sqrt{6}}); \,{x}\in[{-\frac{\pi}{2}};0]\\
    	{x}&=2arctg({\sqrt{3}-\sqrt{6}}); \,{x}\in[{-\frac{\pi}{2}};0]\\
    	\end{aligned}\right
    	\end{aligned}
    \end{equation}\\
   На интервале [0 ; $\frac{5{\pi}}{6}$] график функции ${h(x)}$ не пересекает ось ОХ. \\
      %в точках с абциссами, определяемыми уравнением ${h(x)}=0$, то есть в точках:\\
      %${x}&=2arctg({\sqrt{3}}-{\sqrt{2}});$\\
      %${x}&=2arctg({\sqrt{3}}-{\sqrt{6}}).$
        
      \textbf{4. Промежутки знакопостоянства}\\
     Если функция положительна на интервале - график расположен выше оси абсцисс; если функция отрицательна - график ниже оси абсцисс. \\[10pt]
     $h{(0)} =\sqrt{3}sin(0) + cos(0)$–$(cos(2\cdot{0} + {\frac{\pi}{3})-1})=$\\
     $={0}+{1}$–$cos({\frac{\pi}{3}})+1={\frac{3}{2}}> 0;$\\
     $h{({\frac{5\pi}{6}})} =\sqrt{3}sin({\frac{5\pi}{6}}) + cos({\frac{5\pi}{6}})$–$(cos(2\cdot{{\frac{5\pi}{6}}} + {\frac{\pi}{3})-1})=$\\
     $=\sqrt{3}\cdot{\frac{\sqrt{3}}{2}}$- ${\frac{\sqrt{3}}{2}}$–$1+1={\frac{3-{\sqrt{3}}}{2}}> 0$\\[10pt]
     \textbf{Таким образом,} функция ${h(x)}$ положительна на интервале [0 ; $\frac{5{\pi}}{6}$],следовательно, график расположен выше оси абсцисс.\\
     
      \textbf{5. Промежутки возрастания и убывания функции.}\\
      Если ${f'(x)}>0$ на промежутке, то ${f(x)}$ возрастает на этом промежутке.\\
      Если ${f'(x)}<0$ на промежутке, то ${f(x)}$ убывает на этом промежутке.\\[10pt]
      Первая производная ${h'(x)}$ от функции ${h(x)}$ (с учетом замены на {t}) имеет вид:\\
      \begin{aligned}
      (2\sqrt{3}t&-2t)(3+2\sqrt{3}t-t^2)+(2\sqrt{3}-2t)(2\sqrt{3}t+1-t^2)=0\\
      (2\sqrt{3}t&-2t)(3+2\sqrt{3}t-t^2+2\sqrt{3}t+1-t^2)=0\\
      (2\sqrt{3}t&-2t)(4-2t^2+4\sqrt{3}t)=0
      \end{aligned}\\
      \begin{equation}\label{eq:num18}
      \begin{align}
      2(\sqrt{3}t&-t)2(2-t^2+2\sqrt{3}t)=0
      \end{align}
      \end{equation}
      Найдем корни уравнения \eqref{eq:num18}, для этого решим систему:\\
      \begin{equation}\label{eq:num19}
      \begin{align}
      \left\{
      \begin{aligned}
      &t(\sqrt{3}-1)=0\\
      & 2-t^2+2\sqrt{3}t=0
      \end{aligned}\right
      \end{align}
      \end{equation}
      Корни уравнений:\\
      \begin{equation}\label{eq:num20}
         \left\{
      \begin{aligned}
      {t_1}&={0}\\
      	{t_2}&={\sqrt{3}}-{\sqrt{5}}\\
      	{t_3}&={\sqrt{3}}+{\sqrt{5}}
      \end{aligned}\right
      	 	\end{equation}    
      	 	\\  
      	\begin{equation}\label{eq:num21}
      	\left\{
      	\begin{aligned}
      	tg{\frac{x}{2}}&={0}\\
      			tg{\frac{x}{2}}&={\sqrt{3}}-{\sqrt{5}}\\
      					tg{\frac{x}{2}}&={\sqrt{3}}+{\sqrt{5}}
      									\end{aligned}\right
      									\end{equation}\\
      \begin{equation}\label{eq:num22}
      \left\{
      \begin{aligned}
      {x}&=0;\,{x}\in[{0;\frac{5\pi}{6}}]\\
      {x}&=2arctg({\sqrt{3}-\sqrt{5}}); \,{x}\in[{-\frac{\pi}{2}};0]\\	{x}&=2arctg({\sqrt{3}+\sqrt{5}}); \,{x}\in[{-\frac{\pi}{2}};0]      
      \end{aligned}\right
       \end{equation}
      %\begin{equation}
      \begin{align*}
       {h'(x)}=2[\sqrt{3}tg({\frac{x}{2}})-tg({\frac{x}{2}})]\cdot2[2-tg^2({\frac{x}{2}})+2\sqrt{3}tg({\frac{x}{2}})]=0\hspace{150}
      \end{align*}
     %\end{equation}\\
     \begin{equation}\label{eq:num23}
     \begin{align}
     {h'({0})}=0 \hspace{320}
     \end{align}
     \end{equation}\\
     При $[-\infty;0]$ производная ${h'(x)}<0$, следовательно функция ${h(x)}$ убывает на промежутке [0 ; $\frac{5{\pi}}{6}$].\\
     При $[0;+\infty]$ производная ${h'(x)}>0$, следовательно функция ${h(x)}$ возрастает на промежутке [0 ; $\frac{5{\pi}}{6}$].\\
     
      \textbf{6. Выпуклость графика функции, точки перегиба}\\
     Функция ${f(x)}$ выпукла вниз, если ${f'(x)}$ возрастает на промежутке, \\
     при этом  ${f"(x)}>0$.\\
      Функция ${f(x)}$ выпукла вверх, если ${f'(x)}$ убывает на промежутке, \\
      при этом  ${f"(x)}<0$.\\[10pt]
     Вторая производная ${h"(x)}$ от функции ${h(x)}$ (с учетом замены на {t}) имеет вид:\\
     \begin{equation}\label{eq:num24}
     \begin{align}
     h'(x)=2(\sqrt{3}t-t)\cdot2(2-t^2+2\sqrt{3}t)=0
     \end{align}
     \end{equation}
      \begin{equation}\label{eq:num25}
      \begin{align}
      h'(x)=8\sqrt{3}t-4\sqrt{3}t^3+24t^2-8t+4t^3-8\sqrt{3}t^2
      \end{align}
      \end{equation}
       \begin{equation}\label{eq:num26}
       \begin{align}
       h"(x)=8\sqrt{3}-12\sqrt{3}t^2+48t-8+12t-16\sqrt{3}t
       \end{align}
       \end{equation}
       $\sqrt{D}<0$, значит точек перегиба нет, $h"(x)$ не обращается в ноль.
       \begin{equation}\label{eq:num27}
       \begin{align}
       h"(0)=8\sqrt{3}-8\,\,>0
       \end{align}
       \end{equation}
       \textbf{Таким образом,} $h"(x)>0$ и ${h'(x)}$ возрастает на промежутке, значит
       функция ${h(x)}$ выпукла вниз.\\
       График функции ${h(x)}$ показан на рисунке 1. \\\\
       \begin{center}
       	\includegraphics[scale=2]{graph}
       \end{center}
       \begin{center}
       	Рисунок 1 - График функции ${h(x)}$
       \end{center}
      \newpage
      \subsection{Нахождение коэффициентов кубического сплайна}
     \upshape
      \subsubsection{Задания и исходные данные для решения} 
      $ $1. Найти коэффициенты кубического сплайна, интерполирующего данные, представленные в векторах$\,\,  {\vec{V}_x} \,\,$и$\,\, {\vec{V}_y.}$ \\
      $ {\,\,\,\,\,\,\,\,\,\,\,\,\,\,\,\,\,\,}$2. Построить на одном графике: функцию$\,\, {f(x)}\,\, $и$\,\,  функцию {f_1(x)}, $полученную после нахождения коэффициентов кубического сплайна.$ $ \\
      $ {\,\,\,\,\,\,\,\,\,\,\,\,\,\,\,\,\,\,}$3. Представить графическое изображение результатов интерполяции исходных данных$ $.\\
                       
      $\vec{V}_x=\left(\begin{array}{c}0\\0.5\\1.4\\2.25\\3.5\end{array}\right),
      \,\,\,\vec{V}_y=\left(\begin{array}{c}3.0\\2.7\\3.7\\3.333\\3.667\end{array}\right)$ \\\\
      $ {\,\,\,\,\,\,\,\,\,\,\,\,\,\,\,\,\,\,}$Необходимо оценить погрешность в точке $ {x = 2.4}. $\,\,\,\,\,Вычислить значение функции в точке $\,{x = 1.2}.$\\
      \newpage
      \newpage
      \subsubsection{Теория и вывод уравнения сплайна}
      Уравнение сплайна находится по пяти точкам\\
      $(x_1;y_1), (x_2;y_2), (x_3;y_3), (x_4;y_4), (x_5;y_5)$\\
      Представим сплайн полиномом третьей степени на каждом отрезке
      $[x_i, x_{i+1}]$.
      
      \begin{equation}\label{eq:F_i(x)}
      F_i(x)=A_{i0}+{A_{i1}}x+{A_{i2}}x^2+{A_{i3}}x^3,
      \end{equation}
      
\      $где $x$ \in {\,} $[{x_i},{x_{i+1}}].$\\[4pt]
      Найдем коэффициенты $A_{ij}$ исходя из того, что в точках склейки функция не имеет разрывов, изломов и изгиб ее слева и справа совпадает. \\
      На каждом из отрезков $[x_i, x_{i+1}]$ график $F_i(x)$ проходит через точки $y_i$, $y_{i+1}.$
      \begin{equation}\label{eq:y_i}
      y_i=A_{i0}+{A_{i1}}{x_i}+{A_{i2}}{x_i}^2+{A_{i3}}{x_i}^3
      \end{equation}
            
      Получаем $8$ уравнений:
      \begin{equation}\label{eq:y1(x)}
      \begin{aligned}
      y_1=A_{10}+{A_{11}}{x_1}+{A_{12}}{{x_1}^2}+{A_{13}}{x_1}^3\\[4pt]
      y_2=A_{10}+{A_{11}}{x_2}+{A_{12}}{x_2}^2+{A_{13}}{x_2}^3\\[4pt]
      y_2=A_{20}+{A_{21}}{x_2}+{A_{22}}{x_2}^2+{A_{23}}{x_2}^3\\[4pt]
      y_3=A_{20}+{A_{21}}{x_3}+{A_{22}}{x_3}^2+{A_{23}}{x_3}^3\\[4pt]
      y_3=A_{30}+{A_{31}}{x_3}+{A_{32}}{x_3}^2+{A_{33}}{x_3}^3\\[4pt]
      y_4=A_{30}+{A_{31}}{x_4}+{A_{32}}{x_4}^2+{A_{33}}{x_4}^3\\[4pt]
      y_4=A_{40}+{A_{41}}{x_4}+{A_{42}}{x_4}^2+{A_{43}}{x_4}^3\\[4pt]
      y_5=A_{40}+{A_{41}}{x_5}+{A_{42}}{x_5}^2+{A_{43}}{x_5}^3\\[4pt]
      \end{aligned}
      \end{equation}
       Производные первого порядка во внутренних точках ${x_i}$ должны совпадать,
       т.е. производная слева 
       $${{F_i}^{'}}({x_i}) = A_{i1}+ 2{A_{i2}}{x_i}+ 3A_{i3}{{x_i}^2}$$
        должна быть равна производной справа 
        $${{F^{'}}_{(i+1)}}({x_i}) = {A_{{(i+1)}1}}+ 2{A_{{(i+1)}2}}{x_i}+ 3A_{{(i+1)}3}{{x_i}^2}$$
      Физический смысл равенства производных состоит в том, что в точках склейки у нас нет излома сплайна.     
    \begin{equation}\label{eq:y^{'}(x)}
    \begin{aligned}
    A_{11}+ 2A_{12}{x_2}+ 3A_{13}{{x_2}^2}=A_{21}+ 2A_{22}{x_2}+ 3A_{23}{{x_2}^2}\\
    A_{21}+ 2A_{22}{x_3}+ 3A_{23}{{x_3}^2}=A_{31}+ 2A_{32}{x_3}+ 3A_{33}{{x_3}^2}\\
    A_{31}+ 2A_{32}{x_4}+ 3A_{33}{{x_4}^2}=A_{41}+ 2A_{42}{x_4}+ 3A_{43}{{x_4}^2}\\ 
    \end{aligned}
    \end{equation}
    
      Производные второго порядка в точках склейки ${x_i}$ должны совпадать,
      т.е. вторая производная слева 
      $${{F_i}^{''}}{(x_i)} = 2{A_{i2}}+ 6{A_{i3}}{x_i}$$
      должна быть равна второй производной справа 
      $${{F^{''}}_{(i+1)}{(x_i)} =2{A_{{(i+1)}2}+ 6{A_{{(i+1)}3}}{x_i}$$
         Физический смысл равенства вторых производных состоит в том, что в точках склейки изгиб сплайна справа и слева должен быть одинаковым.
            
      		\begin{equation}\label{eq:y^{'}(x)}
      		\begin{aligned}
      		2{A_{12}}+ 6{A_{13}}{x_2}=2{A_{22}}+ 6{A_{23}}{x_2}\\
      		2{A_{22}}+ 6{A_{23}}{x_3}=2{A_{32}}+ 6{A_{33}}{x_3}\\
      		2{A_{32}}+ 6{A_{33}}{x_4}=2{A_{42}}+ 6{A_{43}}{x_4}\\ 
      		\end{aligned}
      		\end{equation}
      		
      		Еще два уравнения - из граничных условий в крайних точках $x_1$, $x_n$:
      	\begin{equation}\label{eq:F^{''}(x)=0}
      	\begin{aligned}
      	{C_{11}}{F^{'}}{x_1}+{C_{12}}+ {{F^{''}}{(x_1)}={C_{13}}\\
      	C_{n1}{F^{'}}{n_1}+C_{n2}+ {{F^{''}}{(n_2)}=C_{n3}\\ 
      	\end{aligned}
      	\end{equation}
      						
      Найдем график сплайна в случае, когда концы сплайна оставлены
   	свободными в граничных точках $(x1, y1)$, $(x5, y5)$. Соответственно, уравнения имеют вид:
      	\begin{equation}\label{eq:F^{''}(x)}
      	\begin{aligned}
      	2A_{12}+ 6A_{13}{{x_1}=0\\
      	2A_{42}+ 6A_{43}{{x_5}=0\\
      	\end{aligned}
      	\end{equation}
      	В итоге - 16 уравнений для определения 16 коэффициэнтов $A_{ij}$ .	\\
      	
      		%\mathds{A} = \left(
      	\\
      		{\tiny
      		
      		\left(\begin{array}{cccccccccccccccc} 
      			1&{x_1}&{x_1}^2&{x_1}^3&0&0&0&0&0&0&0&0&0&0&0&0\\
      			1&{x_2}&{x_2}^2&{x_2}^3&0&0&0&0&0&0&0&0&0&0&0&0\\
      			0&1&2{x_2}&3{x_2}^2&0&-1&-2{x_2}&-3{x_2}^2&0&0&0&0&0&0&0&0\\
      		    0&0&2&6{x_2}&0&0&-2&-6{x_2}&0&0&0&0&0&0&0&0\\
      		    0&0&0&0&1&{x_2}&{x_2}^2&{x_2}^3&0&0&0&0&0&0&0&0\\
      		    0&0&0&0&1&{x_3}&{x_3}^2&{x_3}^3&0&0&0&0&0&0&0&0\\
      		    0&0&0&0&0&1&2{x_3}&3{x_3}^2&0&-1&-2{x_3}&-3{x_3}^2&0&0&0&0\\
      		    0&0&0&0&0&0&2&6{x_3}&0&0&-2&-6{x_3}&0&0&0&0\\
      		    0&0&0&0&0&0&0&0&1&{x_3}&{x_3}^2&{x_3}^3&0&0&0&0\\
      		    0&0&0&0&0&0&0&0&1&{x_4}&{x_4}^2&{x_4}^3&0&0&0&0\\
      		    0&0&0&0&0&0&0&0&0&1&2{x_4}&3{x_4}^2&0&-1&-2{x_4}&-3{x_4}^2\\
      		    0&0&0&0&0&0&0&0&0&0&2&6{x_4}&0&0&-2&-6{x_4}\\
      		    0&0&0&0&0&0&0&0&0&0&0&0&1&{x_4}&{x_4}^2&{x_4}^3\\
      		    0&0&0&0&0&0&0&0&0&0&0&0&1&{x_5}&{x_5}^2&{x_5}^3\\
      		    0&0&2&6{x_1}&0&0&0&0&0&0&0&0&0&0&0&0\\
      		    0&0&0&0&0&0&0&0&0&0&0&0&0&0&2&6{x_5}
      		    \end{array}\right)
      	$x$
      	\left(\begin{array}{c} 
      		$A_{10}$\\$A_{11}$\\$A_{12}$\\	$A_{13}$\\	
      		$A_{20}$\\$A_{21}$\\$A_{22}$\\	$A_{23}$\\ 
      		$A_{30}$\\$A_{31}$\\$A_{32}$\\	$A_{33}$\\ 
      		$A_{40}$\\$A_{41}$\\$A_{42}$\\	$A_{43}$
      	\end{array}\right)
      		$=$
      		\left(\begin{array}{c} 
      			${y_{1}}$\\$y_{2}$\\$0$\\	$0$\\	
      			$y_{2}$\\$y_{3}$\\$0$\\	$0$\\ 
      			$y_{3}$\\$y_{4}$\\$0$\\	$0$\\ 
      			$y_{4}$\\$y_{5}$\\$0$\\	$0$\\ [15pt]
      		  	\end{array}\right)						
      	
      }\\
     \\[10pt]
      	
      	\centering{\normalsize Коэффициенты $A_{ij}:$}\\
      	              
      $$\centering{\normalsize  
      	\begin{vmatrix}
      \large A_{10}\\A_{11}\\ A_{12}\\ A_{13}\\
      A_{20}\\A_{21}\\A_{22}\\A_{23}\\
      A_{30}\\A_{31}\\A_{32}\\A_{33}\\
      A_{40}\\A_{41}\\A_{42}\\A_{43}
      \end{vmatrix}
      \centering{\large =}
       \begin{vmatrix}
       3\\-1.024\\0\\1.695\\
       3.432\\-3.614\\5.18\\-1.759\\
       -4.785\\13.994\\-7.397\\1.236\\
       12.165\\-8.607\\2.648\\-0.252
              \end{vmatrix}}$$
       \newpage
       
       
      {\left \normalsize $Уравнение сплайна имеет вид:$}\\[10pt]
      {\normalsize ${F(x)}$= \left\{ 
      	\begin{aligned}
      	$F_1(x)&=1.695x^3+0x^2-1.024x+3\,\,\,{x}\,{\in}\,\,\, {$[0,\,0.5]$}};\\
      $F_2(x)&=-1.759x^3+5.18x^2-3.614x+3.432 {x}\,{\in}\, {$[0.5,\,1.4]$}};\\
    $F_3(x)&=1.236x^3-7.397x^2+13.994x-4.785,$\,\, $где$ \,\,\,{x}\,{\in}\,\,\, {$[1.4,\,2.25]$}};\\
$F_4(x)&=-0.252x^3+2.648x^2-8.607x+12.165$,\,\, $где$\,{x}\,{\in}\, {$[2.25,\,3.5]$}}.\\
\end{aligned} 	\right.\]
}
       \\[20pt]
   ${F(1.2)}=19.579$
    \\[100pt]
    $

\\
\begin{tikzpicture}
\begin{scope}[scale=1.5]
\hspace{50}
\begin{axis}
$\addplot[domain=0:0.5,smooth] {(1.695)*x^3 + (0)*x^2 + (-1.024)*x+3};
$\addplot[domain=0.5:1.4,smooth] {(-1.759)*x^3 + (5.18)*x^2 + (-3.614)*x+(3.432)};
\addplot[domain=1.4:2.25,smooth] {(1.236)*x^3 + (-7.397)*x^2 + (13.994)*x+(-4.785)};
\addplot[domain=2.25:3.5,smooth] {(-0.252)*x^3 + (2.648)*x^2 + (-8.607)*x+(12.165)};
\end{axis}
\end{scope}
\end{tikzpicture}
      	  \newpage
      \subsection{Решение задачи оптимального распределения неоднородных ресурсов}
      \textbf{Постановка задачи. }\\
.      Для изготовления ${n}$ видов изделий {И_1}$, {И_2}, \,\,$\cdot \cdot \cdot \,\,${И_n}$ необходимы ресурсы ${m}$ видов: трудовые, материальные, финансовые и др.\\ 
      Известно требуемое количество отдельного $i$-гo ресурса для изготовления каждого $j$-го изделия - норма расхода ${с_ij}$. \\
      Пусть определено количество каждого вида ресурса, которым предприятие располагает в данный момент - ${а_j}$. Известна прибыль $П_j$, получаемая предприятием от изготовления каждого ${j}$-го изделия. \\
      Требуется определить, какие изделия и в каком количестве должны производиться предприятием, чтобы прибыль была максимальной.\\
      Математическая модель задачи выглядит следующим образом.\\
      Целевая функция имеет вид:\\
      25{x_1}+45{x_2}+60{x_3}+20{x_4}$\rightarrow$ max\\
      $Ограничения имеют вид$:$\\
      \begin{equation}
      \left\
      \begin{aligned}
      2{x_1}&+4{x_2}+2{x_3}+9{x_4}=20 \\
      5{x_1}&+5{x_2}+5{x_3}+6{x_4}=10 \\
      5{x_1}&+6{x_2}+4{x_3}+8{x_4}=30 \\
      {x_j}&\geq{0};\,\,{j=\overline{1,4}}
      \end{aligned} \right\}\hspace{270}
      \end{equation}
      \\[10pt]
      После решения системы видно, что\\
      {x_1}={12.5},\\ {x_2}={0},\\{x_3}={7.5},\\ {x_4}={0}. \\
      $В оптимальном решении Изделие1=12.5; Изделие2=0; Изделие3=7.5; Изделие4=0. При этом максимальная прибыль будет составлять 100, а количество использованных ресурсов равно: трудовых=20, материальных=7, финансовых=30.$ 
       	\newpage
      	\section{ Заключение}
      	Курсовая работа состояла из трех частей: \\В первой части была исследована функция и построен ее график. Во второй части найдены коэффициенты кубического сплайна. В третьей части решалась задача оптимального распределения неоднородных ресурсов.
     	\end{document}